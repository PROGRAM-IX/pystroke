%
% API Documentation for pystroke
% Module pystroke.vector2
%
% Generated by epydoc 3.0.1
% [Wed Apr  3 19:10:01 2013]
%

%%%%%%%%%%%%%%%%%%%%%%%%%%%%%%%%%%%%%%%%%%%%%%%%%%%%%%%%%%%%%%%%%%%%%%%%%%%
%%                          Module Description                           %%
%%%%%%%%%%%%%%%%%%%%%%%%%%%%%%%%%%%%%%%%%%%%%%%%%%%%%%%%%%%%%%%%%%%%%%%%%%%

    \index{pystroke \textit{(package)}!pystroke.vector2 \textit{(module)}|(}
\section{Module pystroke.vector2}

    \label{pystroke:vector2}

%%%%%%%%%%%%%%%%%%%%%%%%%%%%%%%%%%%%%%%%%%%%%%%%%%%%%%%%%%%%%%%%%%%%%%%%%%%
%%                               Variables                               %%
%%%%%%%%%%%%%%%%%%%%%%%%%%%%%%%%%%%%%%%%%%%%%%%%%%%%%%%%%%%%%%%%%%%%%%%%%%%

  \subsection{Variables}

    \vspace{-1cm}
\hspace{\varindent}\begin{longtable}{|p{\varnamewidth}|p{\vardescrwidth}|l}
\cline{1-2}
\cline{1-2} \centering \textbf{Name} & \centering \textbf{Description}& \\
\cline{1-2}
\endhead\cline{1-2}\multicolumn{3}{r}{\small\textit{continued on next page}}\\\endfoot\cline{1-2}
\endlastfoot\raggedright \_\-\_\-p\-a\-c\-k\-a\-g\-e\-\_\-\_\- & \raggedright \textbf{Value:} 
{\tt \texttt{'}\texttt{pystroke}\texttt{'}}&\\
\cline{1-2}
\end{longtable}


%%%%%%%%%%%%%%%%%%%%%%%%%%%%%%%%%%%%%%%%%%%%%%%%%%%%%%%%%%%%%%%%%%%%%%%%%%%
%%                           Class Description                           %%
%%%%%%%%%%%%%%%%%%%%%%%%%%%%%%%%%%%%%%%%%%%%%%%%%%%%%%%%%%%%%%%%%%%%%%%%%%%

    \index{pystroke \textit{(package)}!pystroke.vector2 \textit{(module)}!pystroke.vector2.Vector2 \textit{(class)}|(}
\subsection{Class Vector2}

    \label{pystroke:vector2:Vector2}
A two-dimensional vector

\textbf{Author:} James Heslin (PROGRAM\_IX)




%%%%%%%%%%%%%%%%%%%%%%%%%%%%%%%%%%%%%%%%%%%%%%%%%%%%%%%%%%%%%%%%%%%%%%%%%%%
%%                                Methods                                %%
%%%%%%%%%%%%%%%%%%%%%%%%%%%%%%%%%%%%%%%%%%%%%%%%%%%%%%%%%%%%%%%%%%%%%%%%%%%

  \subsubsection{Methods}

    \label{pystroke:vector2:Vector2:__init__}
    \index{pystroke \textit{(package)}!pystroke.vector2 \textit{(module)}!pystroke.vector2.Vector2 \textit{(class)}!pystroke.vector2.Vector2.\_\_init\_\_ \textit{(method)}}

    \vspace{0.5ex}

\hspace{.8\funcindent}\begin{boxedminipage}{\funcwidth}

    \raggedright \textbf{\_\_init\_\_}(\textit{self}, \textit{x}={\tt 0.0}, \textit{y}={\tt 0.0})

    \vspace{-1.5ex}

    \rule{\textwidth}{0.5\fboxrule}
\setlength{\parskip}{2ex}
    Constructs a new Vector2

\setlength{\parskip}{1ex}
      \textbf{Parameters}
      \vspace{-1ex}

      \begin{quote}
        \begin{Ventry}{x}

          \item[x]

          X (horizontal) co-ordinate of vector

            {\it (type=double)}

          \item[y]

          Y (vertical) co-ordinate of vector

            {\it (type=double)}

        \end{Ventry}

      \end{quote}

\textbf{Author:} James Heslin (PROGRAM\_IX)



    \end{boxedminipage}

    \label{pystroke:vector2:Vector2:__str__}
    \index{pystroke \textit{(package)}!pystroke.vector2 \textit{(module)}!pystroke.vector2.Vector2 \textit{(class)}!pystroke.vector2.Vector2.\_\_str\_\_ \textit{(method)}}

    \vspace{0.5ex}

\hspace{.8\funcindent}\begin{boxedminipage}{\funcwidth}

    \raggedright \textbf{\_\_str\_\_}(\textit{self})

    \vspace{-1.5ex}

    \rule{\textwidth}{0.5\fboxrule}
\setlength{\parskip}{2ex}
    Returns a string with the vector's co-ordinates

\setlength{\parskip}{1ex}
      \textbf{Return Value}
    \vspace{-1ex}

      \begin{quote}
      A string containing the vector's co-ordinates

      {\it (type=string)}

      \end{quote}

\textbf{Author:} James Heslin (PROGRAM\_IX)



    \end{boxedminipage}

    \label{pystroke:vector2:Vector2:from_points}
    \index{pystroke \textit{(package)}!pystroke.vector2 \textit{(module)}!pystroke.vector2.Vector2 \textit{(class)}!pystroke.vector2.Vector2.from\_points \textit{(static method)}}

    \vspace{0.5ex}

\hspace{.8\funcindent}\begin{boxedminipage}{\funcwidth}

    \raggedright \textbf{from\_points}(\textit{a}, \textit{b})

    \vspace{-1.5ex}

    \rule{\textwidth}{0.5\fboxrule}
\setlength{\parskip}{2ex}
    Returns a new Vector2 with the co-ordinates of the difference between 
    the two points

\setlength{\parskip}{1ex}
      \textbf{Parameters}
      \vspace{-1ex}

      \begin{quote}
        \begin{Ventry}{x}

          \item[a]

          The first point to use in constructing the new Vector2

            {\it (type=tuple/list of two ints)}

          \item[b]

          The second point to use in constructing the new Vector2

            {\it (type=tuple/list of two ints)}

        \end{Ventry}

      \end{quote}

      \textbf{Return Value}
    \vspace{-1ex}

      \begin{quote}
      A new Vector2 constructed from the inputted points

      {\it (type=Vector2)}

      \end{quote}

\textbf{Author:} James Heslin (PROGRAM\_IX)



    \end{boxedminipage}

    \label{pystroke:vector2:Vector2:get_magnitude}
    \index{pystroke \textit{(package)}!pystroke.vector2 \textit{(module)}!pystroke.vector2.Vector2 \textit{(class)}!pystroke.vector2.Vector2.get\_magnitude \textit{(method)}}

    \vspace{0.5ex}

\hspace{.8\funcindent}\begin{boxedminipage}{\funcwidth}

    \raggedright \textbf{get\_magnitude}(\textit{self})

    \vspace{-1.5ex}

    \rule{\textwidth}{0.5\fboxrule}
\setlength{\parskip}{2ex}
    Returns the magnitude of the vector

\setlength{\parskip}{1ex}
      \textbf{Return Value}
    \vspace{-1ex}

      \begin{quote}
      The magnitude of the vector

      {\it (type=double)}

      \end{quote}

\textbf{Author:} James Heslin (PROGRAM\_IX)



    \end{boxedminipage}

    \label{pystroke:vector2:Vector2:normalised}
    \index{pystroke \textit{(package)}!pystroke.vector2 \textit{(module)}!pystroke.vector2.Vector2 \textit{(class)}!pystroke.vector2.Vector2.normalised \textit{(method)}}

    \vspace{0.5ex}

\hspace{.8\funcindent}\begin{boxedminipage}{\funcwidth}

    \raggedright \textbf{normalised}(\textit{self})

    \vspace{-1.5ex}

    \rule{\textwidth}{0.5\fboxrule}
\setlength{\parskip}{2ex}
    Returns a normalised copy of the vector

\setlength{\parskip}{1ex}
      \textbf{Return Value}
    \vspace{-1ex}

      \begin{quote}
      Normalised copy of the vector

      {\it (type=Vector2)}

      \end{quote}

\textbf{Author:} James Heslin (PROGRAM\_IX)



    \end{boxedminipage}

    \label{pystroke:vector2:Vector2:dot_product}
    \index{pystroke \textit{(package)}!pystroke.vector2 \textit{(module)}!pystroke.vector2.Vector2 \textit{(class)}!pystroke.vector2.Vector2.dot\_product \textit{(method)}}

    \vspace{0.5ex}

\hspace{.8\funcindent}\begin{boxedminipage}{\funcwidth}

    \raggedright \textbf{dot\_product}(\textit{self}, \textit{other})

    \vspace{-1.5ex}

    \rule{\textwidth}{0.5\fboxrule}
\setlength{\parskip}{2ex}
    Returns the dot product of the vector and the input vector

\setlength{\parskip}{1ex}
      \textbf{Parameters}
      \vspace{-1ex}

      \begin{quote}
        \begin{Ventry}{xxxxx}

          \item[other]

          The vector to dot product against

            {\it (type=Vector2)}

        \end{Ventry}

      \end{quote}

      \textbf{Return Value}
    \vspace{-1ex}

      \begin{quote}
      The dot product of the vector and the input vector

      {\it (type=double)}

      \end{quote}

\textbf{Author:} James Heslin (PROGRAM\_IX)



    \end{boxedminipage}

    \label{pystroke:vector2:Vector2:cross_product}
    \index{pystroke \textit{(package)}!pystroke.vector2 \textit{(module)}!pystroke.vector2.Vector2 \textit{(class)}!pystroke.vector2.Vector2.cross\_product \textit{(method)}}

    \vspace{0.5ex}

\hspace{.8\funcindent}\begin{boxedminipage}{\funcwidth}

    \raggedright \textbf{cross\_product}(\textit{self}, \textit{other})

    \vspace{-1.5ex}

    \rule{\textwidth}{0.5\fboxrule}
\setlength{\parskip}{2ex}
    Returns the cross product of the vector and the input vector

\setlength{\parskip}{1ex}
      \textbf{Parameters}
      \vspace{-1ex}

      \begin{quote}
        \begin{Ventry}{xxxxx}

          \item[other]

          The vector to cross product against

            {\it (type=Vector2)}

        \end{Ventry}

      \end{quote}

      \textbf{Return Value}
    \vspace{-1ex}

      \begin{quote}
      The cross product of the vector and the input vector

      {\it (type=double)}

      \end{quote}

\textbf{Author:} James Heslin (PROGRAM\_IX)



    \end{boxedminipage}

    \label{pystroke:vector2:Vector2:clamp}
    \index{pystroke \textit{(package)}!pystroke.vector2 \textit{(module)}!pystroke.vector2.Vector2 \textit{(class)}!pystroke.vector2.Vector2.clamp \textit{(static method)}}

    \vspace{0.5ex}

\hspace{.8\funcindent}\begin{boxedminipage}{\funcwidth}

    \raggedright \textbf{clamp}(\textit{x}, \textit{a}, \textit{b})

    \vspace{-1.5ex}

    \rule{\textwidth}{0.5\fboxrule}
\setlength{\parskip}{2ex}
    'Clamp' the value of x between a and b, i.e., return x if it is between
    a and b, a if x is lower than a, and b if x is larger than b

\setlength{\parskip}{1ex}
      \textbf{Parameters}
      \vspace{-1ex}

      \begin{quote}
        \begin{Ventry}{x}

          \item[x]

          The number to clamp

            {\it (type=double)}

          \item[a]

          The lower bound of x's clamp

            {\it (type=double)}

          \item[b]

          The upper bound of x's clamp

            {\it (type=double)}

        \end{Ventry}

      \end{quote}

      \textbf{Return Value}
    \vspace{-1ex}

      \begin{quote}
      The clamped value of x

      {\it (type=double)}

      \end{quote}

\textbf{Author:} James Heslin (PROGRAM\_IX)



    \end{boxedminipage}

    \label{pystroke:vector2:Vector2:radians_between}
    \index{pystroke \textit{(package)}!pystroke.vector2 \textit{(module)}!pystroke.vector2.Vector2 \textit{(class)}!pystroke.vector2.Vector2.radians\_between \textit{(method)}}

    \vspace{0.5ex}

\hspace{.8\funcindent}\begin{boxedminipage}{\funcwidth}

    \raggedright \textbf{radians\_between}(\textit{self}, \textit{other})

    \vspace{-1.5ex}

    \rule{\textwidth}{0.5\fboxrule}
\setlength{\parskip}{2ex}
    Return the radians between the vector and the input vector

\setlength{\parskip}{1ex}
      \textbf{Parameters}
      \vspace{-1ex}

      \begin{quote}
        \begin{Ventry}{xxxxx}

          \item[other]

          The other vector making the angle

            {\it (type=Vector2)}

        \end{Ventry}

      \end{quote}

      \textbf{Return Value}
    \vspace{-1ex}

      \begin{quote}
      The number of radians between the vector and the input vector

      TODO: Determine if this actually works, it's not being used

      {\it (type=double)}

      \end{quote}

\textbf{Author:} James Heslin (PROGRAM\_IX)



    \end{boxedminipage}

    \label{pystroke:vector2:Vector2:get_angle}
    \index{pystroke \textit{(package)}!pystroke.vector2 \textit{(module)}!pystroke.vector2.Vector2 \textit{(class)}!pystroke.vector2.Vector2.get\_angle \textit{(method)}}

    \vspace{0.5ex}

\hspace{.8\funcindent}\begin{boxedminipage}{\funcwidth}

    \raggedright \textbf{get\_angle}(\textit{self})

    \vspace{-1.5ex}

    \rule{\textwidth}{0.5\fboxrule}
\setlength{\parskip}{2ex}
    Returns the angle this vector is pointing to

\setlength{\parskip}{1ex}
      \textbf{Return Value}
    \vspace{-1ex}

      \begin{quote}
      The angle this vector points to (in radians)

      {\it (type=double)}

      \end{quote}

\textbf{Author:} James Heslin (PROGRAM\_IX)



    \end{boxedminipage}

    \label{pystroke:vector2:Vector2:__add__}
    \index{pystroke \textit{(package)}!pystroke.vector2 \textit{(module)}!pystroke.vector2.Vector2 \textit{(class)}!pystroke.vector2.Vector2.\_\_add\_\_ \textit{(method)}}

    \vspace{0.5ex}

\hspace{.8\funcindent}\begin{boxedminipage}{\funcwidth}

    \raggedright \textbf{\_\_add\_\_}(\textit{self}, \textit{other})

    \vspace{-1.5ex}

    \rule{\textwidth}{0.5\fboxrule}
\setlength{\parskip}{2ex}
    Add the vector to other and return the result

\setlength{\parskip}{1ex}
      \textbf{Parameters}
      \vspace{-1ex}

      \begin{quote}
        \begin{Ventry}{xxxxx}

          \item[other]

          The vector to add

            {\it (type=Vector2)}

        \end{Ventry}

      \end{quote}

      \textbf{Return Value}
    \vspace{-1ex}

      \begin{quote}
      The result of the vector being added to other

      {\it (type=Vector2)}

      \end{quote}

\textbf{Author:} James Heslin (PROGRAM\_IX)



    \end{boxedminipage}

    \label{pystroke:vector2:Vector2:__sub__}
    \index{pystroke \textit{(package)}!pystroke.vector2 \textit{(module)}!pystroke.vector2.Vector2 \textit{(class)}!pystroke.vector2.Vector2.\_\_sub\_\_ \textit{(method)}}

    \vspace{0.5ex}

\hspace{.8\funcindent}\begin{boxedminipage}{\funcwidth}

    \raggedright \textbf{\_\_sub\_\_}(\textit{self}, \textit{other})

    \vspace{-1.5ex}

    \rule{\textwidth}{0.5\fboxrule}
\setlength{\parskip}{2ex}
    Subtract other from the vector and return the result

\setlength{\parskip}{1ex}
      \textbf{Parameters}
      \vspace{-1ex}

      \begin{quote}
        \begin{Ventry}{xxxxx}

          \item[other]

          The vector to subtract

            {\it (type=Vector2)}

        \end{Ventry}

      \end{quote}

      \textbf{Return Value}
    \vspace{-1ex}

      \begin{quote}
      The result of other being subtracted from the vector

      {\it (type=Vector2)}

      \end{quote}

\textbf{Author:} James Heslin (PROGRAM\_IX)



    \end{boxedminipage}

    \label{pystroke:vector2:Vector2:__neg__}
    \index{pystroke \textit{(package)}!pystroke.vector2 \textit{(module)}!pystroke.vector2.Vector2 \textit{(class)}!pystroke.vector2.Vector2.\_\_neg\_\_ \textit{(method)}}

    \vspace{0.5ex}

\hspace{.8\funcindent}\begin{boxedminipage}{\funcwidth}

    \raggedright \textbf{\_\_neg\_\_}(\textit{self})

    \vspace{-1.5ex}

    \rule{\textwidth}{0.5\fboxrule}
\setlength{\parskip}{2ex}
    Negate the vector and return the result

\setlength{\parskip}{1ex}
      \textbf{Return Value}
    \vspace{-1ex}

      \begin{quote}
      The negated vector

      {\it (type=Vector2)}

      \end{quote}

\textbf{Author:} James Heslin (PROGRAM\_IX)



    \end{boxedminipage}

    \label{pystroke:vector2:Vector2:__mul__}
    \index{pystroke \textit{(package)}!pystroke.vector2 \textit{(module)}!pystroke.vector2.Vector2 \textit{(class)}!pystroke.vector2.Vector2.\_\_mul\_\_ \textit{(method)}}

    \vspace{0.5ex}

\hspace{.8\funcindent}\begin{boxedminipage}{\funcwidth}

    \raggedright \textbf{\_\_mul\_\_}(\textit{self}, \textit{sca})

    \vspace{-1.5ex}

    \rule{\textwidth}{0.5\fboxrule}
\setlength{\parskip}{2ex}
    Multiply the vector by other and return the result

\setlength{\parskip}{1ex}
      \textbf{Parameters}
      \vspace{-1ex}

      \begin{quote}
        \begin{Ventry}{xxx}

          \item[sca]

          The scalar to multiply by

            {\it (type=double)}

        \end{Ventry}

      \end{quote}

      \textbf{Return Value}
    \vspace{-1ex}

      \begin{quote}
      The result of the vector being multiplied by sca

      {\it (type=Vector2)}

      \end{quote}

\textbf{Author:} James Heslin (PROGRAM\_IX)



    \end{boxedminipage}

    \label{pystroke:vector2:Vector2:__div__}
    \index{pystroke \textit{(package)}!pystroke.vector2 \textit{(module)}!pystroke.vector2.Vector2 \textit{(class)}!pystroke.vector2.Vector2.\_\_div\_\_ \textit{(method)}}

    \vspace{0.5ex}

\hspace{.8\funcindent}\begin{boxedminipage}{\funcwidth}

    \raggedright \textbf{\_\_div\_\_}(\textit{self}, \textit{sca})

    \vspace{-1.5ex}

    \rule{\textwidth}{0.5\fboxrule}
\setlength{\parskip}{2ex}
    Divide the vector by sca and return the result

\setlength{\parskip}{1ex}
      \textbf{Parameters}
      \vspace{-1ex}

      \begin{quote}
        \begin{Ventry}{xxx}

          \item[sca]

          The scalar to divide by

            {\it (type=double)}

        \end{Ventry}

      \end{quote}

      \textbf{Return Value}
    \vspace{-1ex}

      \begin{quote}
      The result of the vector being divided by sca

      {\it (type=Vector2)}

      \end{quote}

\textbf{Author:} James Heslin (PROGRAM\_IX)



    \end{boxedminipage}

    \index{pystroke \textit{(package)}!pystroke.vector2 \textit{(module)}!pystroke.vector2.Vector2 \textit{(class)}|)}
    \index{pystroke \textit{(package)}!pystroke.vector2 \textit{(module)}|)}
